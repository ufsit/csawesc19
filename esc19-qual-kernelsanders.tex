\documentclass[conference]{IEEEtran}

\pagestyle{plain}

\usepackage{cite}
\usepackage{amsmath,amssymb,amsfonts}
\usepackage{algorithmic}
\usepackage{graphicx}
\usepackage{textcomp}
\usepackage{xcolor}
\usepackage{xspace}
\usepackage{listings}
\usepackage{lmodern}

% (2) specify encoding
\usepackage[T1]{fontenc}

% (3) load symbol definitions
\usepackage{hyperref}

%%%%%%%%%%%%%%%%%%%%%%%%%%%%%%%%%%%%%%%%%%%%%%%%%%%%%%%%%%%%%%%%%%%%%

% For autoref naming of sections
\renewcommand{\sectionautorefname}{Section}
\renewcommand{\subsectionautorefname}{Section}
\renewcommand{\subsubsectionautorefname}{Section}

% Note/todo commands
\newcommand{\needref}{\textcolor{blue}{[ref?]}}
\newcommand\grant[1]{{\color{purple}[Grant: #1]}}

% Listing configuration
\definecolor{comments}{RGB}{74,131,31}
\definecolor{strings}{RGB}{0,128,64}
\definecolor{numbers}{RGB}{44,45,211}
\definecolor{identifiers}{RGB}{153,153,0}

\lstset{
  language=C,
  numbers=none,
  basicstyle={\tt\small},
  stringstyle=\color{strings},
  commentstyle=\color{comments},
  keywordstyle={\color{blue}\bfseries},
  %identifierstyle={\ttfamily\color{identifiers}},
  emph      = [1]{
    challengeFunction,secretFunction},
  emphstyle=[1]{\ttfamily\bfseries\color{blue}},
  stepnumber=1,                   % the step between two line-numbers.
  numbersep=10pt,                  % how far the line-numbers are from the code
  backgroundcolor=\color{white},  % choose the background color. You must add \usepackage{color}
  showspaces=false,               % show spaces adding particular underscores
  showstringspaces=false,         % underline spaces within strings
  showtabs=false,                 % show tabs within strings adding particular underscores
  tabsize=2,                      % sets default tabsize to 2 spaces
  captionpos=b,                   % sets the caption-position to bottom
  breaklines=true,                % sets automatic line breaking
  breakatwhitespace=true,         % sets if automatic breaks should only happen at whitespace
  xleftmargin=2em,framexleftmargin=1.5em
}%

\lstdefinelanguage{none}{
  identifierstyle=
}

%%%%%%%%%%%%%%%%%%%%%%%%%%%%%%%%%%%%%%%%%%%%%%%%%%%%%%%%%%%%%%%%%%%%%

\begin{document}

\title{Kernel Sanders: CSAW {ESC'19} Quals\\
}

\author{\IEEEauthorblockN{
Grant Hernandez\IEEEauthorrefmark{1},
Claire Seiler\IEEEauthorrefmark{1},
Owen Flannagan\IEEEauthorrefmark{1},
Hunter Searle\IEEEauthorrefmark{1},
Kevin R.B. Butler\IEEEauthorrefmark{1}}
\IEEEauthorblockA{\IEEEauthorrefmark{1}University of Florida, Gainesville, FL, USA\\ \{grant.hernandez,
  cseiler, owenflannagan, huntersearle, butler\}@ufl.edu}
}

\maketitle

%\begin{abstract}
%\end{abstract}

\section{Introduction}

The advent of Radio Frequency Identification (RFID) technology has quickly been followed by its rapid and pervasive integration across multiple industries, from transportation to security. However, with its widespread adoption comes an increased need to investigate the potential security implications of integrating RFID technology. As RFID is commonly utilized for a wide variety of authentication, access control, asset tracking, payment, and identification applications, these systems could be vulnerable to attacks that exploit the underlying RFID technology. 

RFID systems typically consist of three main components: an RFID tag, an RFID reader, and an antenna. A reader, which is a two-way radio transmitter-receiver, sends radio frequency signals to tags and reads the response. Tags, which store data like serial numbers, can be read-only or read/write. Additionally, the tags may also be designated as passive or active, meaning they are powered by the radio energy transmitted by the reader or by an on-board battery, respectively.

These systems raise a variety of security concerns; they are potentially vulnerable to a variety of eavesdropping, spoofing, or jamming methods. Additionally, common reverse engineering and firmware exploitation techniques can be applied to the exploitation of RFID readers. 

RFID readers, like other embedded devices, contain firmware in non-volatile, flash memory. This firmware can be extracted from a physical memory chip using tools like \textit{flashrom}, \textit{binwalk}, Bus Pirates, and logic analyzers. Alternatively, if available, JTAG/SWD could be used to perform a memory dump of the running CPU if accessing the firmware via NOR/NAND flash is too difficult. After extraction, static and dynamic analysis can be done to identify potential vulnerabilities that could lead to compromise of the reader. In static analysis, disassemblers like GHIDRA, IDA Pro, radare2, or Binary Ninja can be leveraged to analyze the assembly instructions corresponding to the firmware. A decompiler can assist in understanding and to recreate the source code in a high level language.

Once disassembled or decompiled, specific vulnerabilities can be identified and targeted exploits can be developed. Vulnerabilities like stack- and heap-based buffer overflows, off-by-one errors, integer overflows, uncontrolled format strings, poor input validation and sanitization, OS command injection, disabled (but not removed) debugging functionality, and hardcoded credentials often plague embedded firmware, which typically rely on languages with manual memory management like C or C++. Thus, these types of vulnerabilities can serve as a guide to analyzing the disassembled firmware of an RFID reader. 

After identifying a vulnerability, a targeted exploit can be created. Mitigations like stack canaries, heap protection, ARM's specific eXecute Never (XN), RELocation Read-Only (RELRO), Position-Independent Executable (PIE), and Address Space Layout Randomization (ASLR) can be overcome with some ingenuity and techniques like return-oriented programming (ROP) chaining, stack smashing, heap spraying, information disclosures and more. Exploit writing frameworks, such as pwntools, can assist in developing an exploit for an RFID reader's firmware, depending on the device architecture and the protections enabled.


\section{Challenge}
To begin our analysis of the given \texttt{qualification.out} object, we start by running the GNU \texttt{file} command on it.
\begin{lstlisting}[numbers=left,language=none]
qualification.out: ELF 64-bit LSB executable, x86-64, ... , not stripped
\end{lstlisting}
Immediately we know that this is an x86-64 ELF binary executable, which is unstripped, meaning functions should have names.
Next running \texttt{strings} on the binary (``...'' means snipped text) we see:
\begin{lstlisting}[numbers=left,language=none]
...
Great Job! The flag is what you entered
The flag is <<shhimhiding>>
;*3$"
GCC: (Ubuntu 4.8.4-2ubuntu1~14.04.4) 4.8.4
...
qualification.cpp
...
_Z14secretFunctionv
...
_Z17challengeFunctionPc
\end{lstlisting}
From the strings, we see a ``good flag'' message, an actual flag, that this binary was written as C++, and two C++ mangled functions.

With initial static analysis out of the way, we can set the file as executable and do some dynamic analysis.

\begin{lstlisting}[language=none]
$ chmod +x qualification.out
$ ./qualification.out
$ ./qualification.out test
$ ./qualification.out shhimhiding
\end{lstlisting}

Running the binary with and without arguments (even the flag found via strings) yields no ``goodboy'' message. To investigate further, we start GHIDRA 9.0 to begin our analysis.
We create a new GHIDRA project and load the binary into it. We open the CodeBrowser tool and perform auto-analysis.

The first step in solving this challenge was to look at the main function. This is a simple function that checks if exactly 2 arguments were passed to the program, then calls \texttt{challengeFunction} that takes a char* as it's only parameter. Ghidra outputs the following for \texttt{challengeFunction}.

\begin{lstlisting}
void challengeFunction(char *param_1)
{
	bool bVar1;
	int local_2c;
	uint local_28 [4];
	undefined4 local_18;
	undefined4 local_14;
	undefined4 local_10;
	undefined4 local_c;
	
	local_28[0] = 1;
	local_28[1] = 2;
	local_28[2] = 1;
	local_28[3] = 2;
	local_18 = 1;
	local_14 = 2;
	local_10 = 1;
	local_c = 2;
	bVar1 = true;
	local_2c = 0;
	while (local_2c < 8) {
		if (((int)param_1[(long)local_2c] - 0x30U ^ 3) != local_28[(long)local_2c]) {
			bVar1 = false;
		}
		local_2c = local_2c + 1;
	}
	if (bVar1) {
		puts("Great Job! The flag is what you entered");
	}
	return;
}
\end{lstlisting}

After all the definitions and initialization, the important part of this function is in the while loop. The loop iterates through each of the first 8 chars of the input, applies a simple transformation, then compares it to the corresponding indices of the array, local\textunderscore 28. If each comparison is true, the function prints out a success message. Otherwise, it exits. In order to figure out what input was required, we worked backwards from the local variable. The first 4 numbers in the array are 1, 2, 1, and 2, which are explicitly assigned to the first 4 indices of local\textunderscore 28. Because the array is only allocated with a size of 4, the last 4 comparisons in the while loop run off the end of the array. Space for local variables is allocated on the stack, so the 4 memory spaces immediately after local\textunderscore 28 are the next 4 local variables allocated, namely local\textunderscore 18, local\textunderscore 14, local\textunderscore 10, and local\textunderscore c, with values 1, 2, 1, and 2, respectively. So, after applying the transformation on the input, the first 8 chars must be equal to 1, 2, 1, 2, 1, 2, 1, and 2. The last step is to reverse the transformation, which consists of subtracting the hex value 30, the XORing with 3. The XOR operation turns a 1 into a 2, and a 2 into a 1. Adding 0x30 gives the numerical value of our input as 0x32, 0x31, 0x32, 0x31, 0x32, 0x31, 0x32, and 0x31. Consulting an ASCII table gives the char value for this sequence as "21212121". Running the program with that argument prints out the success message.

Based off of our reverse engineering, we can rename variables and change types to the following:
\begin{lstlisting}
void challengeFunction(char *flag) {
  int i;
  uint table [8];
  bool goodFlag;
  
  table[0] = 1;
  table[1] = 2;
  table[2] = 1;
  table[3] = 2;
  table[4] = 1;
  table[5] = 2;
  table[6] = 1;
  table[7] = 2;
  goodFlag = true;
  i = 0;

  while (i < 8) {
    if (((int)flag[(long)i] - 0x30U ^ 3) != table[(long)i]) {
      goodFlag = false;
    }
    i += 1;
  }
  if (goodFlag) {
    puts("Great Job! The flag is what you entered");
  }
  return;
}
\end{lstlisting}

Further investigation of the functions discovered by GHIDRA, we notice one named \texttt{secretFunction}.

\begin{lstlisting}
void secretFunction(void) {
  puts("The flag is <<shhimhiding>>");
  return;
}
\end{lstlisting}

This function is never referenced by the \texttt{main} or \texttt{challengeFunction}, but it was easily discovered through static analysis (GNU \texttt{strings} also revealed the other flag string).


\section{Conclusion}
In this qualifier we used GHIDRA to reverse engineer an unknown binary file to understand how to provide the correct flag and discover any other interesting features. We recovered the correct flag of ``21212121'' and noticed the false, hidden flag of ``shhimhiding''. With the necessary background in reverse engineering we are prepared to tackle the firmware analysis and exploitation of the RFID platform.


%\section*{References}
%
%EXAMPLE: Please number citations consecutively within brackets \cite{b1}.
%
%\begin{thebibliography}{00}
%\bibitem{b1} G. Eason, B. Noble, and I. N. Sneddon, ``On certain integrals of Lipschitz-Hankel type involving products of Bessel functions,'' Phil. Trans. Roy. Soc. London, vol. A247, pp. 529--551, April 1955.
%\end{thebibliography}
%\vspace{12pt}

\end{document}
