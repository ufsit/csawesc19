\documentclass[conference]{IEEEtran}

\pagestyle{plain}

\usepackage{cite}
\usepackage{amsmath,amssymb,amsfonts}
\usepackage{algorithmic}
\usepackage{graphicx}
\usepackage{textcomp}
\usepackage{xcolor}
\usepackage{xspace}
\usepackage{listings}
\usepackage{lmodern}

% (2) specify encoding
\usepackage[T1]{fontenc}

% (3) load symbol definitions
\usepackage{hyperref}

%%%%%%%%%%%%%%%%%%%%%%%%%%%%%%%%%%%%%%%%%%%%%%%%%%%%%%%%%%%%%%%%%%%%%

% For autoref naming of sections
\renewcommand{\sectionautorefname}{Section}
\renewcommand{\subsectionautorefname}{Section}
\renewcommand{\subsubsectionautorefname}{Section}

% Note/todo commands
\newcommand{\needref}{\textcolor{blue}{[ref?]}}
\newcommand\grant[1]{{\color{purple}[Grant: #1]}}

% Listing configuration
\definecolor{comments}{RGB}{74,131,31}
\definecolor{strings}{RGB}{0,128,64}
\definecolor{numbers}{RGB}{44,45,211}
\definecolor{identifiers}{RGB}{153,153,0}

\lstset{
  language=C,
  numbers=none,
  basicstyle={\tt\small},
  stringstyle=\color{strings},
  commentstyle=\color{comments},
  keywordstyle={\color{blue}\bfseries},
  %identifierstyle={\ttfamily\color{identifiers}},
  emph      = [1]{
    challengeFunction,secretFunction},
  emphstyle=[1]{\ttfamily\bfseries\color{blue}},
  stepnumber=1,                   % the step between two line-numbers.
  numbersep=10pt,                  % how far the line-numbers are from the code
  backgroundcolor=\color{white},  % choose the background color. You must add \usepackage{color}
  showspaces=false,               % show spaces adding particular underscores
  showstringspaces=false,         % underline spaces within strings
  showtabs=false,                 % show tabs within strings adding particular underscores
  tabsize=2,                      % sets default tabsize to 2 spaces
  captionpos=b,                   % sets the caption-position to bottom
  breaklines=true,                % sets automatic line breaking
  breakatwhitespace=true,         % sets if automatic breaks should only happen at whitespace
  xleftmargin=2em,framexleftmargin=1.5em
}%

\lstdefinelanguage{none}{
  identifierstyle=
}

%%%%%%%%%%%%%%%%%%%%%%%%%%%%%%%%%%%%%%%%%%%%%%%%%%%%%%%%%%%%%%%%%%%%%

\begin{document}

\title{Kernel Sanders: CSAW {ESC'19} Quals\\
}

\author{\IEEEauthorblockN{
Grant Hernandez\IEEEauthorrefmark{1},
Claire Seiler\IEEEauthorrefmark{1},
Owen Flannagan\IEEEauthorrefmark{1},
Hunter Searle\IEEEauthorrefmark{1},
Kevin R.B. Butler\IEEEauthorrefmark{1}}
\IEEEauthorblockA{\IEEEauthorrefmark{1}University of Florida, Gainesville, FL, USA\\ \{grant.hernandez,
  cseiler, owenflannagan, huntersearle, butler\}@ufl.edu}
}

\maketitle

%\begin{abstract}
%\end{abstract}

\section{Introduction}

The advent of Radio Frequency Identification (RFID) technology has quickly been followed by its rapid and pervasive integration across multiple industries, from transportation to security. However, with its widespread adoption comes an increased need to investigate the potential security implications of integrating RFID technology. As RFID is commonly utilized for a wide variety of authentication, access control, asset tracking, payment, and identification applications, these systems could be vulnerable to attacks that exploit the underlying RFID technology. 

RFID systems typically consist of three main components: an RFID tag, an RFID reader, and an antenna. A reader, which is a two-way radio transmitter-receiver, sends radio frequency signals to tags and reads the response. Tags, which store data like serial numbers, can be read-only or read/write. Additionally, the tags may also be designated as passive or active, meaning they are powered by the radio energy transmitted by the reader or by an on-board battery, respectively.

These systems raise a variety of security concerns; they are potentially vulnerable to a variety of eavesdropping, spoofing, or jamming methods. Additionally, common reverse engineering and firmware exploitation techniques can be applied to the exploitation of RFID readers. 

RFID readers, like other embedded devices, contain firmware in non-volatile memory. This firmware can be extracted from a physical memory chip using tools like \textit{flashrom}, \textit{binwalk}, Bus Pirates, and logic analyzers. After extraction, static and dynamic analysis can be done to identify potential vulnerabilities that can lead to compromise of the reader. In static analysis, disassemblers like Ghidra, IDA, radare2, or Binary Ninja can be leveraged to analyze the assembly instructions corresponding to the firmware. A decompiler can assist in attempting to recreate the source code in a high level language.  

Once disassembled or decompiled, specific vulnerabilities can be identified and targeted exploits can be developed. Vulnerabilities like stack- and heap-based buffer overflows, off-by-one errors, integer overflows, poor input validation and sanitization, OS command injection, disabled (but not removed) debugging functionality, and hardcoded credentials often plague embedded firmware, which typically rely on languages with manual memory management like C or C++. Thus, these types of vulnerabilities can serve as a guide to analyzing the disassembled firmware of an RFID reader. 

Once a vulnerability is identified, 

\section{Challenge}
First, to enable collaboration amongst team members, we set up a \ghidra shared
project on our own server to share reversing progress.
All relevant challenge binaries were ARM-32 Cortex-M4 (TeensyChallengeSetX.ino.elf).
These contained all of the \texttt{challenge\_X} functions that we had to reverse.
The AVR binaries were used for blinking the LEDs, polling the buttons, and displaying to the LCD. It also acted as the I$^{2}$C bus master, with the Teensy acting as a slave with address \texttt{0x01}.
The single most
difficult part of reversing these challenges with \ghidra was the broken RFID
stack variables. \ghidra was unable, for nearly all challenges, to make the
offsets into the RFID data easily apparent.

%TODO WRITE MORE ABOUT ANGR
%TODO MENTION THAT SYMBOLS INCLUDED MADE THIS EASIER
To assist us with the problem of stack addresses, we used \angr, a python framework for analyzing binaries. In order to analyze binaries, \angr lifts the file into Valgrind's VEX intermediate representation (IR), then uses both static and dynamic (``concolic'') analysis. We used \angr to hook all memory reads and writes. If the R/W fell in the range of the RFID
data on the stack, we printed \texttt{CARD READ: XX} where XX was the hex
offset into the card data. This alone saved us from manually counting offsets
across stack frames. Beyond finding the address of card reads, \angr allowed us to automatically solve many of the simpler challenges. More details are shown in \autoref{sec:lounge}.

\vspace{0.5cm}
\noindent The highlights of our report are:
\begin{itemize}
  \item We employed concolic analysis using \angr to avoid reversing as many challenges as possible
  \item We achieved arbitrary code execution on challenge \mbox{D-bounce} (\autoref{sec:bounce})
\end{itemize}

All hashes for our solved challenges are in the Appendix under \autoref{sec:hashes}.
We were able to solve 16/18 challenges for a total of 1810 points.
Our video demo is available at \url{https://drive.google.com/open?id=1Dxu0LSNhNxHRTTTYGJKsosiagBaUCiCX}


% \begin{lstlisting}[language=none]
% $ chmod +x qualification.out
% $ ./qualification.out
% $ ./qualification.out test
% $ ./qualification.out shhimhiding
% \end{lstlisting}

CHALLENGE


\section{Conclusion}
In this qualifier we used GHIDRA to reverse engineer an unknown binary file to understand how to provide the correct flag and discover any other interesting features. We recovered the correct flag of ``21212121'' and noticed the false, hidden flag of ``shhimhiding''. With the necessary background in reverse engineering we are prepared to tackle the firmware analysis and exploitation of the RFID platform.


%\section*{References}
%
%EXAMPLE: Please number citations consecutively within brackets \cite{b1}.
%
%\begin{thebibliography}{00}
%\bibitem{b1} G. Eason, B. Noble, and I. N. Sneddon, ``On certain integrals of Lipschitz-Hankel type involving products of Bessel functions,'' Phil. Trans. Roy. Soc. London, vol. A247, pp. 529--551, April 1955.
%\end{thebibliography}
%\vspace{12pt}

\end{document}
