The first step in solving this challenge was to look at the main function. This is a simple function that checks if exactly 2 arguments were passed to the program, then calls \texttt{challengeFunction} that takes a char* as it's only parameter. Ghidra outputs the following for \texttt{challengeFunction}.

\begin{lstlisting}
void challengeFunction(char *param_1)
{
	bool bVar1;
	int local_2c;
	uint local_28 [4];
	undefined4 local_18;
	undefined4 local_14;
	undefined4 local_10;
	undefined4 local_c;
	
	local_28[0] = 1;
	local_28[1] = 2;
	local_28[2] = 1;
	local_28[3] = 2;
	local_18 = 1;
	local_14 = 2;
	local_10 = 1;
	local_c = 2;
	bVar1 = true;
	local_2c = 0;
	while (local_2c < 8) {
		if (((int)param_1[(long)local_2c] - 0x30U ^ 3) != local_28[(long)local_2c]) {
			bVar1 = false;
		}
		local_2c = local_2c + 1;
	}
	if (bVar1) {
		puts("Great Job! The flag is what you entered");
	}
	return;
}
\end{lstlisting}
After all the definitions and initialization, the important part of this function is in the while loop. The loop iterates through each of the first 8 chars of the input, applies a simple transformation, then compares it to the corresponding indices of the array, local\textunderscore 28. If each comparison is true, the function prints out a success message. Otherwise, it exits. In order to figure out what input was required, we worked backwards from the local variable. The first 4 numbers in the array are 1, 2, 1, and 2, which are explicitly assigned to the first 4 indices of local\textunderscore 28. Because the array is only allocated with a size of 4, the last 4 comparisons in the while loop run off the end of the array. Space for local variables is allocated on the stack, so the 4 memory spaces immediately after local\textunderscore 28 are the next 4 local variables allocated, namely local\textunderscore 18, local\textunderscore 14, local\textunderscore 10, and local\textunderscore c, with values 1, 2, 1, and 2, respectively. So, after applying the transformation on the input, the first 8 chars must be equal to 1, 2, 1, 2, 1, 2, 1, and 2. The last step is to reverse the transformation, which consists of subtracting the hex value 30, the XORing with 3. The XOR operation turns a 1 into a 2, and a 2 into a 1. Adding 0x30 gives the numerical value of our input as 0x32, 0x31, 0x32, 0x31, 0x32, 0x31, 0x32, and 0x31. Consulting an ASCII table gives the char value for this sequence as "21212121". Running the program with that argument prints out the success message.