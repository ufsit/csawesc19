
The advent of Radio Frequency Identification (RFID) technology has quickly been followed by its rapid and pervasive integration across multiple industries, from transportation to security. However, with its widespread adoption comes an increased need to investigate the potential security implications of integrating RFID technology. As RFID is commonly utilized for a wide variety of authentication, access control, asset tracking, payment, and identification applications, these systems could be vulnerable to attacks that exploit the underlying RFID technology. 

RFID systems typically consist of three main components: an RFID tag, an RFID reader, and an antenna. A reader, which is a two-way radio transmitter-receiver, sends radio frequency signals to tags and reads the response. Tags, which store data like serial numbers, can be read-only or read/write. Additionally, the tags may also be designated as passive or active, meaning they are powered by the radio energy transmitted by the reader or by an on-board battery, respectively.

These systems raise a variety of security concerns; they are potentially vulnerable to a variety of eavesdropping, spoofing, or jamming methods. Additionally, common reverse engineering and firmware exploitation techniques can be applied to the exploitation of RFID readers. 

RFID readers, like other embedded devices, contain firmware in non-volatile memory. This firmware can be extracted from a physical memory chip using tools like \textit{flashrom}, \textit{binwalk}, Bus Pirates, and logic analyzers. After extraction, static and dynamic analysis can be done to identify potential vulnerabilities that can lead to compromise of the reader. In static analysis, disassemblers like Ghidra, IDA, radare2, or Binary Ninja can be leveraged to analyze the assembly instructions corresponding to the firmware. A decompiler can assist in attempting to recreate the source code in a high level language.  

Once disassembled or decompiled, specific vulnerabilities can be identified and targeted exploits can be developed. Vulnerabilities like stack- and heap-based buffer overflows, off-by-one errors, integer overflows, uncontrolled format strings, poor input validation and sanitization, OS command injection, disabled (but not removed) debugging functionality, and hardcoded credentials often plague embedded firmware, which typically rely on languages with manual memory management like C or C++. Thus, these types of vulnerabilities can serve as a guide to analyzing the disassembled firmware of an RFID reader. 

After identifying a vulnerability, a targeted exploit can be created. Mitigations like stack canaries, heap protection, Data Execution Prevention (DEP), RELocation Read-Only (RELRO), Position-Independent Executable (PIE), and Address Space Layout Randomization (ASLR) can be overcome with some ingenuity and techniques like return-oriented programming (ROP) chaining, stack smashing, heap spraying, and more. These techniques could be directly applicable to developing an exploit for an RFID reader's firmware, depending on the device architecture and the protections enabled. 