
The advent of Radio Frequency Identification (RFID) technology has quickly been followed by its rapid and pervasive integration across multiple industries, from transportation to security. However, with its widespread adoption comes an increased need to investigate the potential security implications of integrating RFID technology. As RFID is commonly utilized for a wide variety of authentication, access control, asset tracking, payment, and identification applications, these systems could be vulnerable to attacks that exploit the underlying RFID technology. 

RFID systems typically consist of three main components: an RFID tag, an RFID reader, and an antenna. A reader, which is a two-way radio transmitter-receiver, sends signals to tags and reads the response. Tags, which store data like serial numbers, can be read-only or read/write. Additionally, the tags may also be designated as passive or active, meaning they are powered by the radio energy transmitted by the reader or by an on-board battery, respectively.

Common reverse-engineering and firmware exploitation techniques can be applied to the exploitation of RFID readers. 