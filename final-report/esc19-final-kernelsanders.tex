\documentclass[conference]{IEEEtran}

\pagestyle{plain}

\usepackage{cite}
\usepackage{amsmath,amssymb,amsfonts}
\usepackage{algorithmic}
\usepackage{graphicx}
\usepackage{textcomp}
\usepackage{xcolor}
\usepackage{xspace}
\usepackage{listings}
\usepackage{lmodern}
\usepackage{verbatim}

% (2) specify encoding
\usepackage[T1]{fontenc}

% (3) load symbol definitions
\usepackage{hyperref}

%%%%%%%%%%%%%%%%%%%%%%%%%%%%%%%%%%%%%%%%%%%%%%%%%%%%%%%%%%%%%%%%%%%%%

% For autoref naming of sections
\renewcommand{\sectionautorefname}{Section}
\renewcommand{\subsectionautorefname}{Section}
\renewcommand{\subsubsectionautorefname}{Section}

% Note/todo commands
\newcommand{\needref}{\textcolor{blue}{[ref?]}}
\newcommand\grant[1]{{\color{purple}[Grant: #1]}}

\newcommand{\angr}{\textsc{angr}\xspace}
\newcommand{\ghidra}{\textsc{ghidra}\xspace}

% Listing configuration
\definecolor{comments}{RGB}{74,131,31}
\definecolor{strings}{RGB}{0,128,64}
\definecolor{numbers}{RGB}{44,45,211}
\definecolor{identifiers}{RGB}{153,153,0}

\lstset{
  language=C,
  numbers=none,
  basicstyle={\tt\footnotesize},
  stringstyle=\color{strings},
  commentstyle=\color{comments},
  keywordstyle={\color{blue}\bfseries},
  %identifierstyle={\ttfamily\color{identifiers}},
  emph      = [1]{
    challengeFunction,secretFunction},
  emphstyle=[1]{\ttfamily\bfseries\color{blue}},
  stepnumber=1,                   % the step between two line-numbers.
  numbersep=10pt,                  % how far the line-numbers are from the code
  backgroundcolor=\color{white},  % choose the background color. You must add \usepackage{color}
  showspaces=false,               % show spaces adding particular underscores
  showstringspaces=false,         % underline spaces within strings
  showtabs=false,                 % show tabs within strings adding particular underscores
  tabsize=2,                      % sets default tabsize to 2 spaces
  captionpos=b,                   % sets the caption-position to bottom
  breaklines=true,                % sets automatic line breaking
  breakatwhitespace=false,         % sets if automatic breaks should only happen at whitespace
  xleftmargin=0em,framexleftmargin=1.5em
}%

\lstdefinelanguage{none}{
  identifierstyle=
}

%%%%%%%%%%%%%%%%%%%%%%%%%%%%%%%%%%%%%%%%%%%%%%%%%%%%%%%%%%%%%%%%%%%%%

\begin{document}

\title{Kernel Sanders: CSAW {ESC'19} Final Report\\
}

\author{\IEEEauthorblockN{
Grant Hernandez\IEEEauthorrefmark{1},
Hunter Searle\IEEEauthorrefmark{1},
Owen Flannagan\IEEEauthorrefmark{1},
Claire Seiler\IEEEauthorrefmark{1},
Kevin R.B. Butler\IEEEauthorrefmark{1}}
\IEEEauthorblockA{\IEEEauthorrefmark{1}University of Florida, Gainesville, FL, USA\\ \{grant.hernandez,
  huntersearle, owenflannagan, cseiler, butler\}@ufl.edu}
}

\maketitle

\section{Introduction}
To begin our analysis of the given \texttt{qualification.out} object, we start by running the GNU \texttt{file} command on it.
\begin{lstlisting}[numbers=left,language=none]
qualification.out: ELF 64-bit LSB executable, x86-64, ... , not stripped
\end{lstlisting}
Immediately we know that this is an x86-64 ELF binary executable, which is unstripped, meaning functions should have names.
Next running \texttt{strings} on the binary (``...'' means snipped text) we see:
\begin{lstlisting}[numbers=left,language=none]
...
Great Job! The flag is what you entered
The flag is <<shhimhiding>>
;*3$"
GCC: (Ubuntu 4.8.4-2ubuntu1~14.04.4) 4.8.4
...
qualification.cpp
...
_Z14secretFunctionv
...
_Z17challengeFunctionPc
\end{lstlisting}
From the strings, we see a ``good flag'' message, an actual flag, that this binary was written as C++, and two C++ mangled functions.

With initial static analysis out of the way, we can set the file as executable and do some dynamic analysis.

\begin{lstlisting}[language=none]
$ chmod +x qualification.out
$ ./qualification.out
$ ./qualification.out test
$ ./qualification.out shhimhiding
\end{lstlisting}

Running the binary with and without arguments (even the flag found via strings) yields no ``goodboy'' message. To investigate further, we start GHIDRA 9.0 to begin our analysis.
We create a new GHIDRA project and load the binary into it. We open the CodeBrowser tool and perform auto-analysis.


\section{Challenge Set A}
\subsection{Lounge}
\label{sec:lounge}
Lounge was, at first glance, a difficult challenge due to all of the emulated floating point instructions. Further reversing revealed that only two bytes of card data are used to determine the win condition of \texttt{a * b == 0x18af}. This means the keyspace is $2^{16}$ -- an easily bruteforceable amount. To enable us to bruteforce without manually reflashing card data, we used \angr for dynamic analysis. To start, we created an \angr project:

\begin{lstlisting}[language=python]
import angr
proj = angr.Project("A/TeensyChallengeSetA.ino.elf")
\end{lstlisting}

Then we created a \texttt{blank\_state}, disabled symbolic memory, set the starting PC to \texttt{challenge\_0}, and explored until the goodboy or end of the function:

\begin{lstlisting}[language=python]
st = proj.factory.blank_state()
st.regs.pc = proj.loader.main_object.symbols_by_name["_Z11challenge_06packet"].linked_addr
st.options |= set(["ZERO_FILL_UNCONSTRAINED_MEMORY"])
mgr = self.proj.factory.simgr(st)
mgr.use_technique(angr.exploration_techniques.Explorer(find=[0xc21], avoid=[0xc51]))
mgr.run()
\end{lstlisting}

This performs purely concrete execution until an address in the \texttt{find}
or \texttt{avoid} sets is found. In this case, because the RFID data was
assumed to be zero, the SimulationManager ends with one state in the ``avoid'' stash.
This run took exactly 30 seconds, which is quite a slowdown compared to a real
execution environment. This is because \angr interprets VEX Intermediate
Representation (IR) instead of native machine code, in addition to performing
expensive memory and register bookkeeping. This can incur slowdowns of 100 -
1000x, depending on the instructions being emulated. To alievate this slowdown,
\angr provides addtional execution engines, such as Unicorn\needref, which
executes native instructions, to burn through concrete instruction traces.
Unfortunately, \angr's version of Unicorn \emph{does not}\needref support ARM,
preventing this speedup.

With these constraints, it looked as if concrete bruteforcing with \angr would be too expensive. Ironically, switching to symbolic execution let us discover more than one solution to this problem in less than two hours of wall-clock time. Switching to symbolic execution involved investigating which offsets into the card data were being read by the challenge function. To do this, we hooked all memory reads during execution and printed when a read address fell in the range of the RFID card data on the local stack frame:

\begin{lstlisting}[language=python]
# Determined by breakpointing in angr and correlating to the output of
# `debugPrintPacket'
WHITE_CARD_START_ADDR = 0x7fff0000-0xf
WHITE_CARD_SZ = 16*64
WHITE_CARD_END_ADDR = WHITE_CARD_START_ADDR + WHITE_CARD_SZ
BUTTON_OFFSET = WHITE_CARD_START_ADDR + WHITE_CARD_SZ + 48

def print_card_offsets(state):
    expr = state.inspect.mem_read_address
    # the address could be symbolic, so get 'a' solution
    expr_val = state.solver.eval(expr)

    if expr_val >= WHITE_CARD_START_ADDR and expr_val <= WHITE_CARD_END_ADDR:
        offset = expr_val - WHITE_CARD_START_ADDR
        print("CARD READ: %x (%s)" % (offset, str(expr)))
    elif expr_val == BUTTON_OFFSET:
        print("!!!!!! BUTTON READ !!!!!!")

st.inspect.b('mem_read', when=angr.BP_AFTER, action=print_card_offsets)
\end{lstlisting}

The \texttt{WHITE\_CARD\_START\_ADDR} was determined by manual inspection by stepping through execution with \angr. We enable this breakpoint on every challenge we solve going forward.
In this case, the card offsets were \texttt{0x4c} and \texttt{0x4d}. Once we had determined these, we were able to set these offsets as symbolic variables:

\begin{lstlisting}[language=python]
st.memory.store(WHITE_CARD_START_ADDR+0x4c, st.solver.BVS("input1", 8))
st.memory.store(WHITE_CARD_START_ADDR+0x4d, st.solver.BVS("input2", 8))
\end{lstlisting}

These are the only variables in memory that we made symbolic (the \texttt{ZERO\_FILL\_UNCONSTRAINED\_MEMORY} ensures this).
We also track when the button values are read by a challenge function. This offset was determined by looking at the static RFID structure in \ghidra.
Next, to speed up the execution process, we added lightweight parallelism. We executed until we received a found state with the Explorer PathTechnique shown earlier:

\begin{lstlisting}[language=python]
from multiprocessing import Pool, cpu_count

...

def exec_once_lounge(self, state):
    """ Executed in another process """
    mgr = self.proj.factory.simgr(state)
    mgr.run(n=20)
    return [mgr.active, mgr.found]

def join_results(omgr):
    mgr.active += omgr[0]
    mgr.found += omgr[1]

# get some initial paths
mgr.run(n=4)

pool = Pool(processes=cpu_count())

while not mgr.found:
    print(mgr)

    if len(mgr.active) == 0:
        time.sleep(1)
        continue

    active_st = mgr.active.copy()
    mgr.drop(stash='active')

    print("Distributing %d states" % len(active_st))

    for a in active_st:
        pool.apply_async(exec_once_lounge, args=(a,), callback=join_results)
\end{lstlisting}

Running the above code on dual Intel Xeon CPU E5-2630 v4 @ 2.20GHz CPUS with 40 cores total, we were able to find two paths, at which point the execution halted. To help pretty-print the card data table and buttons, we designed a helper that evaluates the symbolic or concrete card data from an execution state:

\begin{lstlisting}[language=python]
def print_table(self, state):
    table = state.solver.eval(state.memory.load(WHITE_CARD_START_ADDR, WHITE_CARD_SZ), cast_to=bytes)
    buttons = state.solver.eval(state.memory.load(BUTTON_OFFSET, 1), cast_to=int)

    arr = []
    for i in range(64):
        arr += [[c for c in table[i*16:(i+1)*16]]]

    output = []
    output += ["#     0  1  2  3  4  5  6  7  8  9  a  b  c  d  e  f"]
    output += ["p = ["]

    for i, row in enumerate(arr):
        eol = "," if i < 63 else "]"
        row = ", ".join([(("0x%02x" % x) if x != 0 else "0") for x in row])
        output += ["     [" + str(row) + ("]%s # %x" % (eol, i))]


    output += ["a = 0x%x" % ((buttons >> 4) & 0xf)]
    output += ["b = 0x%x" % (buttons & 0xf)]

    print("\n".join(output))
\end{lstlisting}

Calling \texttt{print\_table} allows us to create \texttt{sender.py} files by just copying and pasting the result. We also have a mode to directly program a card if \angr is run on the local machine.

As we did not need to reverse engineer this challenge at all, except to find the goodboy and badboy basic block addresses (0xc21 and 0xc51), no discussion is necessary and if this kind of ``lock'' was used in the real world, it would quite ineffective as the key is too small.
The two solutions we found for this challenge are in \autoref{sec:hashes}.

% % % % % % % % % % % % % % % % % % % % % % % % % % % % % % % % % % % % % % % % % 
\subsection{Closet}
Hash: 293f7b60b994512db99836ae7d5bab88b2d0089f90fcf6d51b95b374200dc20f

% % % % % % % % % % % % % % % % % % % % % % % % % % % % % % % % % % % % % % % % % 
\subsection{Cafe}
Hash: 98bc5b1a13fdda3cca488c06ddac0aa5c5449c8a9294a9e5c297806e7faff007

% % % % % % % % % % % % % % % % % % % % % % % % % % % % % % % % % % % % % % % % % 
\subsection{Stairs}
Hash: 396f4b1cdf1cc2e7680f2a8716a18c887cd489e12232e75b6810e9d5e91426c7


\section{Challenge Set B}
\subsection{Mobile}
Hash: 68514b8771e5894c799f540855afbc36ef70db34d274a64a8d4271bc1f188379
This challenge involved a simple algorithm that, based on the values of an array through which it iterated, would select characters from a lookup table (LUT) containing the ASCII alphabet. The most important code is the following: 
\begin{lstlisting}
lut_it = 0;
hash_it = 7;
for(int i = 1; i < 0x1e; i++){
	if (done == 0) {
		if (indexes[i] == indexes[i - 1]) {
			lut_it = lut_it + 1;
		}
		else {
			if ((indexes[i] == 0) || (indexes[i - 1] == 0)) {
				if (indexes[i - 1] != 0) {
					challHashGen[hash_it] = LUT[lut_it + indexes[i - 1] * 3];
					hash_it = hash_it + 1;
				}
				lut_it = 0;
			}
			else {
				done = 1;
			}
		}
	}
}

\end{lstlisting}
In order to select the nth character from the LUT, \texttt{lut\_it} should be $n\%3$ and indexes[i-1] should be $n/3$. To achieve this, we set an array with $n\%3+1$ instances of the value $n/3$, followed by a single zero. This pattern was repeated for each character that was needed to achieve the correct output message. 

\subsection{Dance}
Hash: e631b32e3e493c51e5c2b22d1486d401c76ac83e3910566924bcc51b2157c837

\subsection{Code}
Hash: 372ded6746e45ef7c8ad5a22c5738a4b5aa982da66bc8a426aa1cca830d05af3

\subsection{Blue}
We were unable to solve this challenge.


\section{Challenge Set C}
\subsection{Uno}
To start, the hint for this challenge was ``Is OISC 1337?''. Looking up OISC lead to the \href{https://en.wikipedia.org/wiki/One_instruction_set_computer}{One Instruction Set Computer wikipedia article}. From previous experience, we know that any arbitrary computation could be created with a \texttt{subleq} instruction. This instruction performs a subtraction of two memory operands and branches if their result is less than or equal to zero. Reading, retyping, and renaming in \ghidra confirmed that this was the instruction set being used:

\begin{lstlisting}
int MEM[0x5f];
// each char +3 from "solved ..."
// need to -3 to restore
char * TABLE = "vroyhg#fkdoohqjh#xqr#defghijklm";

// initialize memory
for (int i = 0; i < 0x5f; i++) {
  if (i < 0x30) MEM[i] = (int)RFID[0x200+i];
  else if (i < 0x40) MEM[i] = (int)RFID[0x240+i];
  else MEM[i] = (int)TABLE[i];
}

while ((hashAddr = hashAddrCpy, PREV_PC = PC, -1 < PC && (PC < 0x3d))) {
  PC_1 = PC + 1;
  PC = PC_1;
  CMP1 = MEM[PREV_PC];
  PC = PREV_PC + 2;
  CMP2 = MEM[PC_1];
  PC = PREV_PC + 3;
  BRANCH_PC = MEM[PREV_PC + 2];
  if (CMP1 == -1) break;
  if ((CMP2 == -1) && (hashAddrCpy < 0x1f)) {
    hashAddrCpy = hashAddrCpy + 1;
    challHash[hashAddr] = (char)MEM[CMP1];
  }
  else {
    // SUBLEQ OISC
    MEM[CMP2] = MEM[CMP2] - MEM[CMP1];
    if (MEM[CMP2] < 1) {
      PC = BRANCH_PC;
    }
  }
}
\end{lstlisting}

The trick for this challenge was that the challHash was uninititialized and had to be assigned to by the OISC loop. Luckily, a transformed version of the output was placed into the MEM region at offset 0x40. This was loaded from a fixed table and was just the ``solved'' string with a character shift of +3. Good thing we have instructions dedicated to subtracting!

Next began the task of writing a \texttt{subleq} program to shift and store the string. To aid development, we created two-pass a python assembler (available under solutions/C-und-8/asm.py). Our resulting subleq assembly was the following:

\lstinputlisting[language={[x86masm]Assembler}]{../final/C-uno-8/fix.S}

Compiling this yielded [30, 64, -1, 64, -1, -1, 31, 1, -1, 31, 3, -1, 33, 33, 15, 32, 33, 18, 33, 34, 21, 0, 34, 27, 34, 34, 0, -1, -1, -1, 3, -1, 95, 0, 0] for a total length of 35 words. This was written to card offset 0x200 as bytes and the challenge was solved.
\angr was used to test and debug the concrete solution without reflashing the card for each iteration.

\subsection{Game}
Our first clue in solving this challenge was a function called \texttt{findBestMove}. This suggested to us that the solution would require putting game moves onto the card in order to play against the program. Within the \texttt{findBestMove} function is another function called \texttt{minimax}, which is a common algorithm for finding optimimal moves in simple games. An examination of \texttt{minimax} made it clear that the game being played is tic-tac-toe. The original board state is saved in a variable in the challenge function. The game begins with the program having moved twice (player X), and the keycard (player O) having moved once:

\begin{verbatim}
xx_
_o_
___
\end{verbatim}

Therefore, the keycard moves first after being scanned. It was a simple matter to plan out our moves to ensure that the keycard ties with the program. The sequence of moves were read in starting at offset 0x9c. The move encoding was each byte was a move with the top nibble being the row and the bottom nibble being the column. The moves to tie were [r, c] (0, 2), (1, 0), and (2, 2). This left the board in the state of a tie, leading to the win condition.

\subsection{Break}
This challenge was aptly named. \angr chewed through it.

\begin{lstlisting}[language=python]
st = self._get_start_state(addr, ['SYMBOL_FILL_UNCONSTRAINED_MEMORY'])

mgr = self.proj.factory.simgr(st)
# explore to the goodboy
mgr.explore(find=[0x11b9])

self.print_table(mgr.found[0])
\end{lstlisting}

\noindent The offsets 0x9f and 0xa0 were set to one and the buttons set to a = 0x4, b = 0x6.

\subsection{Recess}
This challenge was \emph{also} aptly named.

\begin{lstlisting}[language=python]
st = self._get_start_state(addr, ['SYMBOL_FILL_UNCONSTRAINED_MEMORY'])⋅
mgr = self.proj.factory.simgr(st)

# explore to the goodboy
mgr.explore(find=[0x1291])

self.print_table(mgr.found[0])
\end{lstlisting}

\noindent The offsets 0xa1 - 0xa4 were set to the string "g00d" to get the solve.


\section{Challenge Set D}
\subsection{Bounce}
Hash: bb8d9065d2656d3ab62ab650b0543fe73844f4df52f5f5d60cc10b31ae6086ac

\section{Challenge Set E}
\subsection{Steel}
This challenge involved MD5 hashing like B-code, with the catch that a hash was performed multiple times to increase ``security''. The input into the hash function is a single byte that is determined by some transformations on card data (easily solved by \angr). Therefore, we must first determine the single byte of input into the MD5 hashing rounds to solve this problem. At first glance, this problem seems like it can be solved with a trivial python bruteforcer:

\begin{lstlisting}[language=python]
import hashlib

target = "703224f765d313ee4ed0fadcf9d63a5e"

for i in range(256):
    obj = hashlib.md5()
    obj.update(chr(i))
    res = obj.hexdigest()

    for i in range(9):
        obj.update(res)
        res = obj.hexdigest()

    if res == target:
        print("FOUND: " + chr(i))
        break
\end{lstlisting}

This proved to be wrong due to the implmentation details of padding during the calls to \texttt{H45H::Final}, which Python's hashlib did not respect. To account for this, we downloaded the library that H45H was compiled from: Hashlib++.\footnote{\url{http://hashlib2plus.sourceforge.net/}}
We wrote the following program to mirror what we saw in \ghidra:

\begin{lstlisting}
unsigned char buff[16] = "";|⋅
std::string target = "703224f765d313ee4ed0fadcf9d63a5e";

for (int i = 0; i < 255; i++) {
  MD5 * md5 = new MD5();
  HL_MD5_CTX ctx;
  unsigned char inp = i;

  memset(&ctx, 0, sizeof(ctx));

  md5->MD5Init(&ctx);
  md5->MD5Update(&ctx, &inp, (unsigned int)1);
  md5->MD5Final((unsigned char *)buff, &ctx);

  std::string hexdigest = convToString(buff);

  for (int j = 0; j < 9; j++) {
    md5->MD5Update(&ctx, (unsigned char *)hexdigest.c_str(), 32);
    md5->MD5Final((unsigned char *)buff, &ctx);

    hexdigest = convToString(buff);
  }

  if (hexdigest == target) {
    std::cout << "Got it: " << inp << std::endl;
    break;
  }

  delete md5;
}
\end{lstlisting}

Compiling and running with \texttt{g++ -I build/include/ -o test test.cpp build/lib/libhl++.a \&\& ./test} still did not find any results. Debugging with \angr to compare the output of the second Final call showed a mismatch. Digging into the Final function source code yielded the answer:

\begin{lstlisting}[numbers=left,xleftmargin=1em]
void MD5::MD5Final (unsigned char digest[16], HL_MD5_CTX *context)
{
   ...

   /*
    * Zeroize sensitive information.
    */
   MD5_memset ((POINTER)context, 0, sizeof (*context));
}
\end{lstlisting}

The memset on line 8 was NOT in the compiled version running on the Teensy. Commenting this line out allowed the test program to find the correct hash input of semicolon (`;'). With this initial input, we could now use \angr to solve the first transforms with the known ending constraint of semicolon:

\begin{lstlisting}[language=python]
st.memory.store(WHITE_CARD_START_ADDR+0x191, st.solver.BVS('input', 8*3))

mgr = self.proj.factory.simgr(st)
mgr.explore(find=[0x1796+1])

s = mgr.found[0]
s.solver.add(s.memory.load(s.regs.r7+0x8c, 1) == ord(';'))

self.print_table(s)
\end{lstlisting}

\subsection{Caeser}
We were unable to solve this challenge due to time constraints.

\subsection{Spiral}
This challenge was easily solved by \angr and did not require any reversing:

\begin{lstlisting}[language=python]
st = self._get_start_state(addr, ['ZERO_FILL_UNCONSTRAINED_MEMORY'])
mgr = self.proj.factory.simgr(st)

st.memory.store(WHITE_CARD_START_ADDR+0x18d, st.solver.BVS("input", 8*4))
mgr.explore(find=[0x1e05], avoid=[0x1e2b])

s = mgr.found[0]
self.print_table(s)
\end{lstlisting}

\subsection{Tower}
Examining the challenge showed it was comparing against a SHA-256 hash again, but this time with an input length of 13. Even if the password was only lower-case letters, this would require more than $2 x 10^{18}$ hashes -- far exceeding a bruteforceable limit. There was a base64 encoded string above the hashing that decoded to `ht'. We assumed this stood for hash table and wasted time looking for one. We also used ocl-hashcat with as many wordlists and rule sets as we could given the time, but no matches were found. Further investigation showed that there were more base64 strings encoded throughout. We collected and decoded them all below:

\begin{lstlisting}[language=python]
parts = ['ht', 'tps', 'geW', '://pas', '.com/', 'Ve', 'in', 'teb', 'mJP']
\end{lstlisting}

We rearrainging them into \url{https://pastebin.com/VegeWmJP}
which led to the password \texttt{ndixlelxivnwl}! We burned this on to the card starting at offset 0x180 and got the solve.


\section{Challenge Set F}
\subsection{Spire}
We were unable to solve this challenge.


\onecolumn
\newpage
\appendix
\section*{Challenge Hashes}
\label{sec:hashes}
\verbatiminput{../final/solution_hashes}

%\section*{References}
%
%EXAMPLE: Please number citations consecutively within brackets \cite{b1}.
%
%\begin{thebibliography}{00}
%\bibitem{b1} G. Eason, B. Noble, and I. N. Sneddon, ``On certain integrals of Lipschitz-Hankel type involving products of Bessel functions,'' Phil. Trans. Roy. Soc. London, vol. A247, pp. 529--551, April 1955.
%\end{thebibliography}
%\vspace{12pt}

\end{document}
