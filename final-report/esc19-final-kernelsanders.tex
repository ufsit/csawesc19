\documentclass[conference]{IEEEtran}

\pagestyle{plain}

\usepackage{cite}
\usepackage{amsmath,amssymb,amsfonts}
\usepackage{algorithmic}
\usepackage{graphicx}
\usepackage{textcomp}
\usepackage{xcolor}
\usepackage{xspace}
\usepackage{listings}
\usepackage{lmodern}
\usepackage{verbatim}

% (2) specify encoding
\usepackage[T1]{fontenc}

% (3) load symbol definitions
\usepackage{hyperref}

%%%%%%%%%%%%%%%%%%%%%%%%%%%%%%%%%%%%%%%%%%%%%%%%%%%%%%%%%%%%%%%%%%%%%

% For autoref naming of sections
\renewcommand{\sectionautorefname}{Section}
\renewcommand{\subsectionautorefname}{Section}
\renewcommand{\subsubsectionautorefname}{Section}

% Note/todo commands
\newcommand{\needref}{\textcolor{blue}{[ref?]}}
\newcommand\grant[1]{{\color{purple}[Grant: #1]}}

\newcommand{\angr}{\textsc{angr}\xspace}
\newcommand{\ghidra}{\textsc{ghidra}\xspace}

% Listing configuration
\definecolor{comments}{RGB}{74,131,31}
\definecolor{strings}{RGB}{0,128,64}
\definecolor{numbers}{RGB}{44,45,211}
\definecolor{identifiers}{RGB}{153,153,0}

\lstset{
  language=C,
  numbers=none,
  basicstyle={\tt\footnotesize},
  stringstyle=\color{strings},
  commentstyle=\color{comments},
  keywordstyle={\color{blue}\bfseries},
  %identifierstyle={\ttfamily\color{identifiers}},
  emph      = [1]{
    challengeFunction,secretFunction},
  emphstyle=[1]{\ttfamily\bfseries\color{blue}},
  stepnumber=1,                   % the step between two line-numbers.
  numbersep=0pt,                  % how far the line-numbers are from the code
  backgroundcolor=\color{white},  % choose the background color. You must add \usepackage{color}
  showspaces=false,               % show spaces adding particular underscores
  showstringspaces=false,         % underline spaces within strings
  showtabs=false,                 % show tabs within strings adding particular underscores
  tabsize=2,                      % sets default tabsize to 2 spaces
  captionpos=b,                   % sets the caption-position to bottom
  breaklines=true,                % sets automatic line breaking
  breakatwhitespace=false,         % sets if automatic breaks should only happen at whitespace
  xleftmargin=0em,framexleftmargin=1.5em
}%

\lstdefinelanguage{none}{
  identifierstyle=
}

%%%%%%%%%%%%%%%%%%%%%%%%%%%%%%%%%%%%%%%%%%%%%%%%%%%%%%%%%%%%%%%%%%%%%

\begin{document}

\title{Kernel Sanders: CSAW {ESC'19} Final Report\\
}

\author{\IEEEauthorblockN{
Grant Hernandez\IEEEauthorrefmark{1},
Hunter Searle\IEEEauthorrefmark{1},
Owen Flannagan\IEEEauthorrefmark{1},
Claire Seiler\IEEEauthorrefmark{1},
Kevin R.B. Butler\IEEEauthorrefmark{1}}
\IEEEauthorblockA{\IEEEauthorrefmark{1}University of Florida, Gainesville, FL, USA\\ \{grant.hernandez,
  huntersearle, owenflannagan, cseiler, butler\}@ufl.edu}
}

\maketitle

\section{Introduction}
First, to enable collaboration amongst team members, we set up a \ghidra shared
project on our own server to share reversing progress.
All relevant challenge binaries were ARM-32 Cortex-M4 (TeensyChallengeSetX.ino.elf).
These contained all of the \texttt{challenge\_X} functions that we had to reverse.
The AVR binaries were used for blinking the LEDs, polling the buttons, and displaying to the LCD. It also acted as the I$^{2}$C bus master, with the Teensy acting as a slave with address \texttt{0x01}.
The single most
difficult part of reversing these challenges with \ghidra was the broken RFID
stack variables. \ghidra was unable, for nearly all challenges, to make the
offsets into the RFID data easily apparent.

%TODO WRITE MORE ABOUT ANGR
%TODO MENTION THAT SYMBOLS INCLUDED MADE THIS EASIER
To assist us with the problem of stack addresses, we used \angr, a python framework for analyzing binaries. In order to analyze binaries, \angr lifts the file into Valgrind's VEX intermediate representation (IR), then uses both static and dynamic (``concolic'') analysis. We used \angr to hook all memory reads and writes. If the R/W fell in the range of the RFID
data on the stack, we printed \texttt{CARD READ: XX} where XX was the hex
offset into the card data. This alone saved us from manually counting offsets
across stack frames. Beyond finding the address of card reads, \angr allowed us to automatically solve many of the simpler challenges. More details are shown in \autoref{sec:lounge}.

\vspace{0.5cm}
\noindent The highlights of our report are:
\begin{itemize}
  \item We employed concolic analysis using \angr to avoid reversing as many challenges as possible
  \item We achieved arbitrary code execution on challenge \mbox{D-bounce} (\autoref{sec:bounce})
\end{itemize}

All hashes for our solved challenges are in the Appendix under \autoref{sec:hashes}.
We were able to solve 16/18 challenges for a total of 1810 points.
Our video demo is available at \url{https://drive.google.com/open?id=1Dxu0LSNhNxHRTTTYGJKsosiagBaUCiCX}


% \begin{lstlisting}[language=none]
% $ chmod +x qualification.out
% $ ./qualification.out
% $ ./qualification.out test
% $ ./qualification.out shhimhiding
% \end{lstlisting}


\section{Challenge Set A}
\subsection{Lounge}
\label{sec:lounge}
Lounge was, at first glance, a difficult challenge due to all of the emulated floating point instructions. Further reversing revealed that only two bytes of card data are used to determine the win condition of \texttt{a * b == 0x18af}. This means the keyspace is $2^{16}$ -- an easily bruteforceable amount. To enable us to bruteforce without manually reflashing card data, we used \angr for dynamic analysis. To start, we created an \angr project:

\begin{lstlisting}[language=python]
import angr
proj = angr.Project("A/TeensyChallengeSetA.ino.elf")
\end{lstlisting}

Then we created a \texttt{blank\_state}, disabled symbolic memory, set the starting PC to \texttt{challenge\_0}, and explored until the goodboy or end of the function:

\begin{lstlisting}[language=python]
st = proj.factory.blank_state()
st.regs.pc = proj.loader.main_object.symbols_by_name["_Z11challenge_06packet"].linked_addr
st.options |= set(["ZERO_FILL_UNCONSTRAINED_MEMORY"])
mgr = self.proj.factory.simgr(st)
mgr.use_technique(angr.exploration_techniques.Explorer(find=[0xc21], avoid=[0xc51]))
mgr.run()
\end{lstlisting}

This performs purely concrete execution until an address in the \texttt{find}
or \texttt{avoid} sets is found. In this case, because the RFID data was
assumed to be zero, the SimulationManager ends with one state in the ``avoid'' stash.
This run took exactly 30 seconds, which is quite a slowdown compared to a real
execution environment. This is because \angr interprets VEX Intermediate
Representation (IR) instead of native machine code, in addition to performing
expensive memory and register bookkeeping. This can incur slowdowns of 100 -
1000x, depending on the instructions being emulated. To alievate this slowdown,
\angr provides addtional execution engines, such as Unicorn\needref, which
executes native instructions, to burn through concrete instruction traces.
Unfortunately, \angr's version of Unicorn \emph{does not}\needref support ARM,
preventing this speedup.

With these constraints, it looked as if concrete bruteforcing with \angr would be too expensive. Ironically, switching to symbolic execution let us discover more than one solution to this problem in less than two hours of wall-clock time. Switching to symbolic execution involved investigating which offsets into the card data were being read by the challenge function. To do this, we hooked all memory reads during execution and printed when a read address fell in the range of the RFID card data on the local stack frame:

\begin{lstlisting}[language=python]
# Determined by breakpointing in angr and correlating to the output of
# `debugPrintPacket'
WHITE_CARD_START_ADDR = 0x7fff0000-0xf
WHITE_CARD_SZ = 16*64
WHITE_CARD_END_ADDR = WHITE_CARD_START_ADDR + WHITE_CARD_SZ
BUTTON_OFFSET = WHITE_CARD_START_ADDR + WHITE_CARD_SZ + 48

def print_card_offsets(state):
    expr = state.inspect.mem_read_address
    # the address could be symbolic, so get 'a' solution
    expr_val = state.solver.eval(expr)

    if expr_val >= WHITE_CARD_START_ADDR and expr_val <= WHITE_CARD_END_ADDR:
        offset = expr_val - WHITE_CARD_START_ADDR
        print("CARD READ: %x (%s)" % (offset, str(expr)))
    elif expr_val == BUTTON_OFFSET:
        print("!!!!!! BUTTON READ !!!!!!")

st.inspect.b('mem_read', when=angr.BP_AFTER, action=print_card_offsets)
\end{lstlisting}

The \texttt{WHITE\_CARD\_START\_ADDR} was determined by manual inspection by stepping through execution with \angr. We enable this breakpoint on every challenge we solve going forward.
In this case, the card offsets were \texttt{0x4c} and \texttt{0x4d}. Once we had determined these, we were able to set these offsets as symbolic variables:

\begin{lstlisting}[language=python]
st.memory.store(WHITE_CARD_START_ADDR+0x4c, st.solver.BVS("input1", 8))
st.memory.store(WHITE_CARD_START_ADDR+0x4d, st.solver.BVS("input2", 8))
\end{lstlisting}

These are the only variables in memory that we made symbolic (the \texttt{ZERO\_FILL\_UNCONSTRAINED\_MEMORY} ensures this).
We also track when the button values are read by a challenge function. This offset was determined by looking at the static RFID structure in \ghidra.
Next, to speed up the execution process, we added lightweight parallelism. We executed until we received a found state with the Explorer PathTechnique shown earlier:

\begin{lstlisting}[language=python]
from multiprocessing import Pool, cpu_count

...

def exec_once_lounge(self, state):
    """ Executed in another process """
    mgr = self.proj.factory.simgr(state)
    mgr.run(n=20)
    return [mgr.active, mgr.found]

def join_results(omgr):
    mgr.active += omgr[0]
    mgr.found += omgr[1]

# get some initial paths
mgr.run(n=4)

pool = Pool(processes=cpu_count())

while not mgr.found:
    print(mgr)

    if len(mgr.active) == 0:
        time.sleep(1)
        continue

    active_st = mgr.active.copy()
    mgr.drop(stash='active')

    print("Distributing %d states" % len(active_st))

    for a in active_st:
        pool.apply_async(exec_once_lounge, args=(a,), callback=join_results)
\end{lstlisting}

Running the above code on dual Intel Xeon CPU E5-2630 v4 @ 2.20GHz CPUS with 40 cores total, we were able to find two paths, at which point the execution halted. To help pretty-print the card data table and buttons, we designed a helper that evaluates the symbolic or concrete card data from an execution state:

\begin{lstlisting}[language=python]
def print_table(self, state):
    table = state.solver.eval(state.memory.load(WHITE_CARD_START_ADDR, WHITE_CARD_SZ), cast_to=bytes)
    buttons = state.solver.eval(state.memory.load(BUTTON_OFFSET, 1), cast_to=int)

    arr = []
    for i in range(64):
        arr += [[c for c in table[i*16:(i+1)*16]]]

    output = []
    output += ["#     0  1  2  3  4  5  6  7  8  9  a  b  c  d  e  f"]
    output += ["p = ["]

    for i, row in enumerate(arr):
        eol = "," if i < 63 else "]"
        row = ", ".join([(("0x%02x" % x) if x != 0 else "0") for x in row])
        output += ["     [" + str(row) + ("]%s # %x" % (eol, i))]


    output += ["a = 0x%x" % ((buttons >> 4) & 0xf)]
    output += ["b = 0x%x" % (buttons & 0xf)]

    print("\n".join(output))
\end{lstlisting}

Calling \texttt{print\_table} allows us to create \texttt{sender.py} files by just copying and pasting the result. We also have a mode to directly program a card if \angr is run on the local machine.

As we did not need to reverse engineer this challenge at all, except to find the goodboy and badboy basic block addresses (0xc21 and 0xc51), no discussion is necessary and if this kind of ``lock'' was used in the real world, it would quite ineffective as the key is too small.
The two solutions we found for this challenge are in \autoref{sec:hashes}.

% % % % % % % % % % % % % % % % % % % % % % % % % % % % % % % % % % % % % % % % % 
\subsection{Closet}
Hash: 293f7b60b994512db99836ae7d5bab88b2d0089f90fcf6d51b95b374200dc20f

% % % % % % % % % % % % % % % % % % % % % % % % % % % % % % % % % % % % % % % % % 
\subsection{Cafe}
Hash: 98bc5b1a13fdda3cca488c06ddac0aa5c5449c8a9294a9e5c297806e7faff007

% % % % % % % % % % % % % % % % % % % % % % % % % % % % % % % % % % % % % % % % % 
\subsection{Stairs}
Hash: 396f4b1cdf1cc2e7680f2a8716a18c887cd489e12232e75b6810e9d5e91426c7


\section{Challenge Set B}
\subsection{Mobile}
Hash: 68514b8771e5894c799f540855afbc36ef70db34d274a64a8d4271bc1f188379
This challenge involved a simple algorithm that, based on the values of an array through which it iterated, would select characters from a lookup table (LUT) containing the ASCII alphabet. The most important code is the following: 
\begin{lstlisting}
lut_it = 0;
hash_it = 7;
for(int i = 1; i < 0x1e; i++){
	if (done == 0) {
		if (indexes[i] == indexes[i - 1]) {
			lut_it = lut_it + 1;
		}
		else {
			if ((indexes[i] == 0) || (indexes[i - 1] == 0)) {
				if (indexes[i - 1] != 0) {
					challHashGen[hash_it] = LUT[lut_it + indexes[i - 1] * 3];
					hash_it = hash_it + 1;
				}
				lut_it = 0;
			}
			else {
				done = 1;
			}
		}
	}
}

\end{lstlisting}
In order to select the nth character from the LUT, lut_it should be $n\%3$ and indexes[i-1] should be $n/3$. To achieve this, we set an array with $n\%3+1$ instances of the value $n/3$, followed by a single zero. This pattern was repeated for each character that was needed to achieve the correct output message. 

\subsection{Dance}
Hash: e631b32e3e493c51e5c2b22d1486d401c76ac83e3910566924bcc51b2157c837

\subsection{Code}
Hash: 372ded6746e45ef7c8ad5a22c5738a4b5aa982da66bc8a426aa1cca830d05af3

\subsection{Blue}
We were unable to solve this challenge.


\section{Challenge Set C}
\subsection{Lounge}
TODO

\subsection{Closet}
TODO

\subsection{Cafe}
TODO

\subsection{Stairs}
TODO


\section{Challenge Set D}
\subsection{Lounge}
TODO

\subsection{Closet}
TODO

\subsection{Cafe}
TODO

\subsection{Stairs}
TODO


\section{Challenge Set E}
\subsection{Steel}
We were unable to solve this challenge.

\subsection{Caeser}
We were unable to solve this challenge.

\subsection{Spiral}
We were unable to solve this challenge.

\subsection{Tower}
We were unable to solve this challenge.


\section{Challenge Set F}
\subsection{Spire}
We were unable to solve this challenge.


\onecolumn
\newpage
\appendix
\section*{Challenge Hashes}
\label{sec:hashes}
\verbatiminput{../final/solution_hashes}

%\section*{References}
%
%EXAMPLE: Please number citations consecutively within brackets \cite{b1}.
%
%\begin{thebibliography}{00}
%\bibitem{b1} G. Eason, B. Noble, and I. N. Sneddon, ``On certain integrals of Lipschitz-Hankel type involving products of Bessel functions,'' Phil. Trans. Roy. Soc. London, vol. A247, pp. 529--551, April 1955.
%\end{thebibliography}
%\vspace{12pt}

\end{document}
