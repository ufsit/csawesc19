First, to enable collaboration amongst team members, we set up a \ghidra shared
project on our own server to share reversing progress.
All relevant challenge binaries were ARM-32 Cortex-M4 (TeensyChallengeSetX.ino.elf).
These contained all of the \texttt{challenge\_X} functions that we had to reverse.
The AVR binaries were used for blinking the LEDs, polling the buttons, and displaying to the LCD. It also acted as the I$^{2}$C bus master, with the Teensy acting as a slave with address \texttt{0x01}.
The single most
difficult part of reversing these challenges with \ghidra was the broken RFID
stack variables. \ghidra was unable, for nearly all challenges, to make the
offsets into the RFID data easily apparent.

%TODO WRITE MORE ABOUT ANGR
%TODO MENTION THAT SYMBOLS INCLUDED MADE THIS EASIER
To assist us with the problem of stack addresses, we used \angr, a python framework for analyzing binaries. In order to analyze binaries, \angr lifts the file into Valgrind's VEX intermediate representation (IR), then uses both static and dynamic ("concolic") analysis. We used \angr to hook all memory reads and writes. If the R/W fell in the range of the RFID
data on the stack, we printed \texttt{CARD READ: XX} where XX was the hex
offset into the card data. This alone saved us from manually counting offsets
across stack frames. Beyond finding the address of card reads, \angr allowed us to automatically solve many of the simpler challenges. More details are shown in \autoref{sec:lounge}.

\vspace{0.5cm}
\noindent The highlights of our report are:
\begin{itemize}
  \item We employed concolic analysis using \angr to avoid reversing as many challenges as possible
  \item We achieved arbitrary code execution on challenge \mbox{D-bounce}
\end{itemize}

All hashes for our solved challenges are in the Appendix under \autoref{sec:hashes}.
We were able to solve TODO/18 challenges for a total of TODO points.


% \begin{lstlisting}[language=none]
% $ chmod +x qualification.out
% $ ./qualification.out
% $ ./qualification.out test
% $ ./qualification.out shhimhiding
% \end{lstlisting}
