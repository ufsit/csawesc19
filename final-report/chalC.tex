\subsection{Uno}
Hash: 4842370a583df2fd6328d4a30b09c6bb58d0df767691872b7263e01aed9651cd

\subsection{Game}
Hash: 63c0b41f89bbf493ba791c092b3e5473e243b9c16666f1e5eaa82bc52eeb1613
Our first clue in solving this challenge was a function called findBestMove. This suggested to us that the solution would require putting game moves onto the card in order to play against the program. Within the findBestMove function is another function called minimax, which is a common algorithm for solving simple games. An examination of minimax made it clear that the game being played is tic-tac-toe. The original board state is saved in a variable in the challenge function. The game begins with the program having moved twice, and the keycard having moved once. Therefore, the keycard moves first after being scanned. It was a simple matter to plan out our moves to ensure that the keycard ties with the program, and thus beating the challenge. 

\subsection{Break}
Hash: ae4be3d07679f53cf7fd0ed9669d06bc3b22c5554c81e3bad04986bb7ab91db1

\subsection{Recess}
Hash: 370815b8d8fde829f5c35f893d0b4139d61a775baa4181fcac1fffe014bde9ea
