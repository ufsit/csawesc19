\subsection{Lounge}
\label{sec:lounge}
Lounge was, at first glance, a difficult challenge due to all of the emulated floating point instructions. Further reversing revealed that only two bytes of card data are used to determine the win condition of \texttt{a * b == 0x18af}. This means the keyspace is $2^{16}$ -- an easily bruteforceable amount. To enable us to bruteforce without manually reflashing card data, we used \angr for dynamic analysis. To start, we created an \angr project:

\begin{lstlisting}[language=python]
import angr
proj = angr.Project("A/TeensyChallengeSetA.ino.elf")
\end{lstlisting}

Then we created a \texttt{blank\_state}, disabled symbolic memory, set the starting PC to \texttt{challenge\_0}, and explored until the goodboy or end of the function:

\begin{lstlisting}[language=python]
st = proj.factory.blank_state()
st.regs.pc = proj.loader.main_object.symbols_by_name["_Z11challenge_06packet"].linked_addr
st.options |= set(["ZERO_FILL_UNCONSTRAINED_MEMORY"])
mgr = self.proj.factory.simgr(st)
mgr.use_technique(angr.exploration_techniques.Explorer(find=[0xc21], avoid=[0xc51]))
mgr.run()
\end{lstlisting}

This performs purely concrete execution until an address in the \texttt{find}
or \texttt{avoid} sets is found. In this case, because the RFID data was
assumed to be zero, the SimulationManager ends with one state in the ``avoid'' stash.
This run took exactly 30 seconds, which is quite a slowdown compared to a real
execution environment. This is because \angr interprets VEX Intermediate
Representation (IR) instead of native machine code, in addition to performing
expensive memory and register bookkeeping. This can incur slowdowns of 100 -
1000x, depending on the instructions being emulated. To alievate this slowdown,
\angr provides addtional execution engines, such as Unicorn\href{http://www.unicorn-engine.org/}, which
executes native instructions, to burn through concrete instruction traces.
Unfortunately, \angr's version of Unicorn \emph{does not} support ARM,
preventing this speedup.
%I couldn't find a reference for Unicorn not supporting ARM

With these constraints, it looked as if concrete bruteforcing with \angr would be too expensive. Ironically, switching to symbolic execution let us discover more than one solution to this problem in less than two hours of wall-clock time. Switching to symbolic execution involved investigating which offsets into the card data were being read by the challenge function. To do this, we hooked all memory reads during execution and printed when a read address fell in the range of the RFID card data on the local stack frame:

\begin{lstlisting}[language=python]
# Determined by breakpointing in angr and correlating to the output of
# `debugPrintPacket'
WHITE_CARD_START_ADDR = 0x7fff0000-0xf
WHITE_CARD_SZ = 16*64
WHITE_CARD_END_ADDR = WHITE_CARD_START_ADDR + WHITE_CARD_SZ
BUTTON_OFFSET = WHITE_CARD_START_ADDR + WHITE_CARD_SZ + 48

def print_card_offsets(state):
    expr = state.inspect.mem_read_address
    # the address could be symbolic, so get 'a' solution
    expr_val = state.solver.eval(expr)

    if expr_val >= WHITE_CARD_START_ADDR and expr_val <= WHITE_CARD_END_ADDR:
        offset = expr_val - WHITE_CARD_START_ADDR
        print("CARD READ: %x (%s)" % (offset, str(expr)))
    elif expr_val == BUTTON_OFFSET:
        print("!!!!!! BUTTON READ !!!!!!")

st.inspect.b('mem_read', when=angr.BP_AFTER, action=print_card_offsets)
\end{lstlisting}

The \texttt{WHITE\_CARD\_START\_ADDR} was determined by manual inspection by stepping through execution with \angr. We enable this breakpoint on every challenge we solve going forward.
In this case, the card offsets were \texttt{0x4c} and \texttt{0x4d}. Once we had determined these, we were able to set these offsets as symbolic variables:

\begin{lstlisting}[language=python]
st.memory.store(WHITE_CARD_START_ADDR+0x4c, st.solver.BVS("input1", 8))
st.memory.store(WHITE_CARD_START_ADDR+0x4d, st.solver.BVS("input2", 8))
\end{lstlisting}

These are the only variables in memory that we made symbolic (the \texttt{ZERO\_FILL\_UNCONSTRAINED\_MEMORY} ensures this).
We also track when the button values are read by a challenge function. This offset was determined by looking at the static RFID structure in \ghidra.
Next, to speed up the execution process, we added lightweight parallelism. We executed until we received a found state with the Explorer PathTechnique shown earlier:

\begin{lstlisting}[language=python]
from multiprocessing import Pool, cpu_count

...

def exec_once_lounge(self, state):
    """ Executed in another process """
    mgr = self.proj.factory.simgr(state)
    mgr.run(n=20)
    return [mgr.active, mgr.found]

def join_results(omgr):
    mgr.active += omgr[0]
    mgr.found += omgr[1]

# get some initial paths
mgr.run(n=4)

pool = Pool(processes=cpu_count())

while not mgr.found:
    print(mgr)

    if len(mgr.active) == 0:
        time.sleep(1)
        continue

    active_st = mgr.active.copy()
    mgr.drop(stash='active')

    print("Distributing %d states" % len(active_st))

    for a in active_st:
        pool.apply_async(exec_once_lounge, args=(a,), callback=join_results)
\end{lstlisting}

Running the above code on dual Intel Xeon CPU E5-2630 v4 @ 2.20GHz CPUS with 40 cores total, we were able to find two paths, at which point the execution halted. To help pretty-print the card data table and buttons, we designed a helper that evaluates the symbolic or concrete card data from an execution state:

\begin{lstlisting}[language=python]
def print_table(self, state):
    table = state.solver.eval(state.memory.load(WHITE_CARD_START_ADDR, WHITE_CARD_SZ), cast_to=bytes)
    buttons = state.solver.eval(state.memory.load(BUTTON_OFFSET, 1), cast_to=int)

    arr = []
    for i in range(64):
        arr += [[c for c in table[i*16:(i+1)*16]]]

    output = []
    output += ["#     0  1  2  3  4  5  6  7  8  9  a  b  c  d  e  f"]
    output += ["p = ["]

    for i, row in enumerate(arr):
        eol = "," if i < 63 else "]"
        row = ", ".join([(("0x%02x" % x) if x != 0 else "0") for x in row])
        output += ["     [" + str(row) + ("]%s # %x" % (eol, i))]


    output += ["a = 0x%x" % ((buttons >> 4) & 0xf)]
    output += ["b = 0x%x" % (buttons & 0xf)]

    print("\n".join(output))
\end{lstlisting}

Calling \texttt{print\_table} allows us to create \texttt{sender.py} files by just copying and pasting the result. We also have a mode to directly program a card if \angr is run on the local machine.

As we did not need to reverse engineer this challenge at all, except to find the goodboy and badboy basic block addresses (0xc21 and 0xc51), no discussion is necessary and if this kind of ``lock'' was used in the real world, it would quite ineffective as the key is too small.
The two solutions we found for this challenge are in \autoref{sec:hashes}.

% % % % % % % % % % % % % % % % % % % % % % % % % % % % % % % % % % % % % % % % % 
\subsection{Closet}
With the basic \angr framework created for the previous challenge, we were able to easily support a new challenge. All that needed to be changed was the initial starting function address. Unlike the previous challenge, symbolic execution with \angr did not fare so well.
We encountered constraint explosion due to a symbolic memory read on line 18 below:

\begin{lstlisting}[numbers=left,xleftmargin=2em]
char table[128];
char key[12] = "ESC19-rocks!"
bool good = true;

for (int i = 0x5c; i < 0x84; i++) {
  if (i < 0x70) {
    // stored in table at +0x6c
    table[i + 0x10] = RFID[i];
    Print::println((Print *)&Serial,i + -0x5c);
  } else if (0x7f < i) {
    table[i] = RFID[i];
    Print::println((Print *)&Serial, i + -0x6c);
  }
}

for (int i = 0; i < 12; i++) {
  // this causes angr to blow up as it is a symbolic index
  if (key[i] != table[(uint)table[i + 0x6c] + 0x6c])
    good = false;
}
...
\end{lstlisting}

Assuming the first \texttt{table} load was symbolic, then the next table load's address would be symbolic. \angr instead of loading from a single address, loads from 256 addresses within the table \emph{simultaneously}. This causes the result of the last table lookup to be the disjunction of 256 separate memory loads. These yield massive constraints which get passed on to the Z3\href{https://github.com/angr/angr-z3} constraint solver, which greatly slows down. The time to determine the satisfiability increases each time through the loop. This ends up taking so long that 99\% of the time executing is spent in the solver. We tried to solve this by preconstraining our symbolic card input to resonable values, but this still causes slow downs and final card data outputs to be wrong. Instead for this challenge, we manually solved it by dumping the concrete stack data in \angr and measuring the offset from the table load to the already in-memory key \texttt{ESC19-rocks!}. This offset was \texttt{0x18+i}. The two lines needed to solve this concretely with \angr are below:

\begin{lstlisting}[language=python]
for i in range(0xc):
   st.memory.store(WHITE_CARD_START_ADDR+0x5c+i, pack("<b", 0x18+i))
\end{lstlisting}

From a security perspective, it should be noted that we are able to read outside the bounds of the \texttt{table} variable. In this case, it was relevant as the key was outside of the table (we did not need to pass it in via the RFID table).

% % % % % % % % % % % % % % % % % % % % % % % % % % % % % % % % % % % % % % % % % 
\subsection{Cafe}
The cafe challenge involved a linear transformation of a template challenge hash with multiple XORs and various logic. Like previous challenges, we worked smart and avoided any reversing and just threw \angr at the challenge. The catch for this challenge versus others is that there is no dependent branch that indicates whether the challenge was solved or not. Instead the template challenge hash is transformed in various ways and MUST equal the string \texttt{solved challenge cafe abcdefg}.

We solved this challenge using \angr in symbolic mode. At this point, we wrapped all our \angr usage in a class with helpers to aid the development. For more information read the \texttt{angresc.py} file included with the challenge submission. The relevant lines from the solver are included below:

\begin{lstlisting}[language=python]
self._set_start_symbol("_Z11challenge_26packet")
addr = self.sym.linked_addr

self._hook_prints()
st = self._get_start_state(addr, ['SYMBOL_FILL_UNCONSTRAINED_MEMORY'])

mgr = self.proj.factory.simgr(st)
# We used Veritesting to aggressively merge states and save execution time
# Without this, execution took much more time
mgr.use_technique(angr.exploration_techniques.Veritesting())

mgr.run()

# The final state becomes unconstrained with it returns from the challenge
# function as the saved LR is left as symbolic (intentionally)
if not mgr.unconstrained:
    print("Analysis failed")
    return

s = mgr.unconstrained[0]

fixed = 'solved challenge cafe abcdefg'

# address of challResult seen in GHIDRA
challResult = 0x1fffa140

# Constrain the hash
for i in range(len(fixed)):
    s.solver.add(s.memory.load(challResult+i, 1) == ord(fixed[i]))

# Eval the string at the address given the constraints
strout = self.read_string(s, challResult)
print("ChallResult: " +  repr(strout))

self.print_table(s)
\end{lstlisting}

The highlights of this challenge are the use of Veritesting\href{https://github.com/angr/angr/blob/master/angr/analyses/veritesting.py} and additional constraints to get the desired output. Veritesting enables aggressive state merging, shoving more responsibility to the solver. At the end of the function only a single symbolic state with all the possible symbolic constraints for each byte in the challenge hash OR'd with each out. This state reaches the end of the function and becomes unconstrained as its saved LR is symbolic to prevent returns from the challenge function. Then using the single unconstrained path, we add additional constraints to the challenge hash variable gleaned from \ghidra to yield the correct card table upon printing.

% % % % % % % % % % % % % % % % % % % % % % % % % % % % % % % % % % % % % % % % % 
\subsection{Stairs}
We were given the solution for stairs, but we solved it with \angr anyways.

\begin{lstlisting}[language=python]
self._set_start_symbol("_Z11challenge_36packet")
addr = self.sym.linked_addr

self._hook_prints()
st = self._get_start_state(addr, ['SYMBOL_FILL_UNCONSTRAINED_MEMORY'])

mgr = self.proj.factory.simgr(st)
mgr.use_technique(angr.exploration_techniques.Explorer(find=[0xf61], avoid=[0xf85,0xf39]))

mgr.run()

s = mgr.found[0]
mgr_final = self.proj.factory.simgr(s)
mgr_final.run()

self.print_table(s)
\end{lstlisting}

\noindent No issues were experienced with solving this with \angr.
