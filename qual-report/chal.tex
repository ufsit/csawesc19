The first step in solving this challenge was to look at the main function. This is a simple function that checks if exactly 2 arguments were passed to the program, then calls \texttt{challengeFunction} that takes a char* as it's only parameter. Ghidra outputs the following for \texttt{challengeFunction}.

\begin{lstlisting}
void challengeFunction(char *param_1)
{
	bool bVar1;
	int local_2c;
	uint local_28 [4];
	undefined4 local_18;
	undefined4 local_14;
	undefined4 local_10;
	undefined4 local_c;
	
	local_28[0] = 1;
	local_28[1] = 2;
	local_28[2] = 1;
	local_28[3] = 2;
	local_18 = 1;
	local_14 = 2;
	local_10 = 1;
	local_c = 2;
	bVar1 = true;
	local_2c = 0;
	while (local_2c < 8) {
		if (((int)param_1[(long)local_2c] - 0x30U ^ 3) != local_28[(long)local_2c]) {
			bVar1 = false;
		}
		local_2c = local_2c + 1;
	}
	if (bVar1) {
		puts("Great Job! The flag is what you entered");
	}
	return;
}
\end{lstlisting}

After all the definitions and initialization, the important part of this function is in the while loop. The loop iterates through each of the first 8 chars of the input, applies a simple transformation, then compares it to the corresponding indices of the array, local\textunderscore 28. If each comparison is true, the function prints out a success message. Otherwise, it exits. In order to figure out what input was required, we worked backwards from the local variable. The first 4 numbers in the array are 1, 2, 1, and 2, which are explicitly assigned to the first 4 indices of local\textunderscore 28. Because the array is only allocated with a size of 4, the last 4 comparisons in the while loop run off the end of the array. Space for local variables is allocated on the stack, so the 4 memory spaces immediately after local\textunderscore 28 are the next 4 local variables allocated, namely local\textunderscore 18, local\textunderscore 14, local\textunderscore 10, and local\textunderscore c, with values 1, 2, 1, and 2, respectively. So, after applying the transformation on the input, the first 8 chars must be equal to 1, 2, 1, 2, 1, 2, 1, and 2. The last step is to reverse the transformation, which consists of subtracting the hex value 30, the XORing with 3. The XOR operation turns a 1 into a 2, and a 2 into a 1. Adding 0x30 gives the numerical value of our input as 0x32, 0x31, 0x32, 0x31, 0x32, 0x31, 0x32, and 0x31. Consulting an ASCII table gives the char value for this sequence as "21212121". Running the program with that argument prints out the success message.

Based off of our reverse engineering, we can rename variables and change types to the following:
\begin{lstlisting}
void challengeFunction(char *flag) {
  int i;
  uint table [8];
  bool goodFlag;
  
  table[0] = 1;
  table[1] = 2;
  table[2] = 1;
  table[3] = 2;
  table[4] = 1;
  table[5] = 2;
  table[6] = 1;
  table[7] = 2;
  goodFlag = true;
  i = 0;

  while (i < 8) {
    if (((int)flag[(long)i] - 0x30U ^ 3) != table[(long)i]) {
      goodFlag = false;
    }
    i += 1;
  }
  if (goodFlag) {
    puts("Great Job! The flag is what you entered");
  }
  return;
}
\end{lstlisting}

\subsection{A deeper look at the assembly}
In order to understand how to reach the code that puts the affirmative message, it is important to understand how to prevent goodFlag from being set to False. As goodFlag is initialized to true, it is necessary to avoid the conditional passing. To better understand this code, we looked at this region as x86 assembly.

\begin{lstlisting}[numbers=left]
LAB_0040057e          XREF[1]:     0x4005b6(j)  
0x40057e 8b45dc    MOV    EAX, dword ptr [RBP + local_2c]
0x400581 4863d0    MOVSXD RDX, EAX
0x400584 488b45c8  MOV    RAX, qword ptr [RBP + local_40]
0x400588 4801d0    ADD    RAX, RDX
0x40058b 0fb600    MOVZX  EAX, byte ptr [RAX]
0x40058e 8845db    MOV    byte ptr [RBP + local_2d], AL
0x400591 0fbe45db  MOVSX  EAX, byte ptr [RBP + local_2d]
0x400595 83e830    SUB    EAX, 0x30
0x400598 83f003    XOR    EAX, 0x3
0x40059b 89c2      MOV    EDX, EAX
0x40059d 8b45dc    MOV    EAX, dword ptr [RBP + local_2c]
0x4005a0 4898      CDQE
0x4005a2 8b4485e0  MOV    EAX, dword ptr [RBP + RAX*0x4 + -0x20]
0x4005a6 39c2      CMP    EDX, EAX
0x4005a8 7404      JZ     LAB_004005ae
0x4005aa c645da00  MOV    byte ptr [RBP + local_2e], 0x0
\end{lstlisting}

In this assembly, \texttt{RBP + local\_2c} holds the value of i that increments from 0 to 8. Additionally, \texttt{RBP + local\_40} holds the parameter that is passed to this \texttt{challengeFunction}, and this is the argument to the program itself. When RAX and RDX are added at 0x400588, this is used to create a pointer to the i'th character of the string, and this character is moved into EAX and \texttt{RBP + local\_2d}. After 0x30 is subtracted from this value, it is xored with 0x3. 0x30 is notable because this is the ascii value for the character '0', so subtracting 0x30 from any character of a one digit integer would retrieve it's value. 

The incrementing value i is moved into EAX again at 0x4005a0, and this time it is multiplied by 0x4 and added to \texttt{RBP - 0x20}. This is where the array of 0's and 1's is stored, and this is statically created at the beginning of the function. When these are compared, execution will jump to 004005ae if they are equal, and 0 is moved into \texttt{RBP + local\_2e} if not. This local variable holds the boolean that we need to remain 1. Luckily, xor is a reversible operation, and addition is as well. 1 xored with 3 is 2, and 2 + 0x30 is 0x32. This is the character '2' in ASCII. 2 xored with 3 is 1, and 1 + 0x30 is 0x31, or '1' in ASCII. Since we know the order in which the values of 1 and 2 are assigned into the static array, we can determine that the argument to give the program is \texttt{21212121}.
Running \texttt{./qualification.out} with the argument of 21212121 gives the affirmative message.

Further investigation of the functions discovered by GHIDRA, we notice one named \texttt{secretFunction}.

\begin{lstlisting}
void secretFunction(void) {
  puts("The flag is <<shhimhiding>>");
  return;
}
\end{lstlisting}

This function is never referenced by the \texttt{main} or \texttt{challengeFunction}, but it was easily discovered through static analysis (GNU \texttt{strings} also revealed the other flag string.
